\documentclass{article}
\usepackage[utf8]{inputenc}
\usepackage{amsmath}
\usepackage{graphicx}
\title{Plan de la Démonstration}
\begin{document}
\maketitle

\section{Introduction}
Nous comptons expliquer brièvement les scripts tels que : exec-decac que nous avons mis en place pour faciliter et dont nous utiliserons pendant la démonstration.\newline
\textbf{Détails :} Ce script permet de compiler et d'exécuter des fichiers .deca facilement, en automatisant le processus de compilation et en affichant les erreurs directement dans le terminal.\newline

\section{Fonctionnement du Compilateur}
\subsection{Tests avec Erreurs}
Montrer le bon fonctionnement du compilateur en faisant passer une série de tests qui contiennent des erreurs générées par le compilateur afin d'observer ces erreurs dans le terminal avant de décommenter le bon code qui est déjà dans chaque fichier test afin de montrer la disparition des erreurs précédentes, et ceci successivement pour l'étape A, B et C.\newline
\textbf{Détails :} Nous allons exécuter des tests spécifiques qui provoquent des erreurs de syntaxe et de sémantique, puis corriger ces erreurs pour démontrer la robustesse du compilateur.\newline

\section{Conformité du Compilateur}
\subsection{Tests Spécifiques}
Pour l'étape C, nous montrerons la conformité du compilateur grâce à 3 tests, le test de defaut_registre.deca, et les autres du même répertoire.\newline
\textbf{Détails :} Ces tests vérifieront si le compilateur respecte les spécifications du langage et gère correctement les cas limites.\newline

\section{Optimisation du Code}
Enfin, nous continuerons par faire des éloges du code généré grâce aux optimisations depuis le test de defaut_registre.deca.\newline
\textbf{Détails :} Nous mettrons en avant les améliorations de performance et de lisibilité du code généré par le compilateur après optimisation.

\section{Capacités de l'Extension}
Nous parlerons des capacités de l'extension grâce au test demo_trigo.deca.

\section{Test Global}
Nous finirons par parler du test global qui teste tout le compilateur qui s'appelle demo_poo.deca ainsi que le jeu morpion du même répertoire.

\end{document}